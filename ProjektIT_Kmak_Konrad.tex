\documentclass[12pt,a4paper]{article}
\usepackage[polish]{babel}
\usepackage[T1]{fontenc}
\usepackage[utf8x]{inputenc}
\usepackage{hyperref}
\usepackage{url}
\usepackage{graphicx}

\addtolength{\hoffset}{-1.5cm}
\addtolength{\marginparwidth}{-1.5cm}
\addtolength{\textwidth}{3cm}
\addtolength{\voffset}{-1cm}
\addtolength{\textheight}{2.5cm}
\setlength{\topmargin}{0cm}
\setlength{\headheight}{0cm}

\title{Dokumentacja projektu}
\author{Kmak Konrad}
\date{\today}

\begin{document}

\maketitle

\newpage

\tableofcontents

\newpage

\section{Wstęp}
\subsection{Cel dokumentacji}

Plan projektu jest niezbędnym elementem w procesie projektowania aplikacji. Nieustannie zmieniająca się sytuacja na rynku wymaga, aby każdy projekt został odpowiednio przygotowany, a także przeanalizowany. 

Celem niniejszego dokumentu jest określenie niezbędnych elementów, pozwalających na utworzenie spójnej, poprawnej, oraz w pełni zgodnej z wymaganiami klienta aplikacji. Dokładne przygotowanie planu, umożliwia dostrzeżenie, a następnie korektę możliwych zagrożeń które mogłyby znacząco wpłynąć na efekt końcowy, skutkując stratami finansowymi, lub brakiem możliwości ukończenia projektu. Dokument ten można traktować jako harmonogram projektu, przedstawiający przybliżony czas realizacji, a także plan pracy obligujący do wykonania wskazanych w nim zadań. 

Plan projektu jest dokumentem formalnym, określającym główne elementy związane z procesem powstawania aplikacji, dzięki czemu może zostać wykorzystany jako dokument potwierdzający jakość wykonanego oprogramowania, a także być podstawą do potwierdzenia, czy klient oraz wykonawca w pełni wywiązali się z warunków zawartych w umowie.

\subsection{Zakres dokumentacji}

Zakres niniejszej dokumentacji przedstawia informacje na temat planowania, analizy, zarządzania oraz kosztów tworzonej aplikacji. Początkowe informacje pozwalają na zapoznanie się z przeznaczeniem dokumentu oraz jego głównym zastosowaniem. Przechodząc do głównej części dokumentacji, można zauważyć że jest w pełni poświęcona procesowi planowania oraz zarządzania aplikacją. Zawarte są w niej takie aspekty jak: zakres projektu, czy model procesu projektowego. W tej części przedstawione są także informacje na temat zarządzania projektem, takie jak: 
\begin{itemize}
    \item podział struktury organizacyjnej,
    \item metody komunikacji, 
    \item cele i priorytety, 
    \item ryzyko i jego kontrola.
\end{itemize}
Końcowa część dokumentacji określa wymagania techniczne oraz harmonogram. Zawarte tam informacje, przybliżają w jakiej metodologii zostanie wykonany projekt, wraz z przedstawieniem narzędzi wykorzystywanych do utworzenia aplikacji. Ostatni punkt planu w pełni poświęcony jest harmonogramowi, który przedstawia etapy projektu, wraz z zadaniami oraz przyporządkowanymi do nich zasobami. Wraz z harmonogramem przedstawiony jest budżet, w którym zaprezentowany jest przewidywany kosz pracy nad aplikacją.

\subsection{Przeznaczenie dokumentu}

Najważniejszą osobą odpowiedzialną za utworzenie, a następnie aktualizację i utrzymanie niniejszej dokumentacji jest kierownik projektu. Kierownik jest wyspecjalizowanym członkiem zespołu, posiadającym szeroką wiedzę z zakresu zarządzania projektami, co czyni go osobą odgrywającą najważniejszą role w procesie powstawania aplikacji. Kierownik wykorzystując swoje doświadczenie zawodowe, a także odpowiednie przygotowanie merytoryczne, buduje zespół, prowadzi rozmowy z osobami związanymi z projektem, a następnie wykonuje dokumentację, która jest elementem niezbędnym do osiągnięcia sukcesu, za który odpowiada w trakcie trwania całego projektu.

Czynnym uczestnikiem w trakcie powstawania dokumentacji jest także zespół wykonawczy, odpowiedzialny za implementację aplikacji. Poprawna komunikacja pomiędzy kierownikiem projektu, a zespołem skutkuje doborem odpowiednich technologii, harmonogramem dostosowanym do możliwości pracowników, a także zmniejszeniem ryzyka przeszacowania czasu projektu.

Osobą mającą wpływ na końcowy wygląd planu projektu, jest również klient. Osoba zamawiająca aplikację, powinna jasno określić swoje wymagania dotyczące wizji projektu, pozwalając na utworzenie dokładnego harmonogramu. Klient sprawdzając w trakcie powstawania dokumentacji, czy jego koncepcja aplikacji pokrywa się z wizją kierownika projektu (którą określił w zakresie planu projektu), daje możliwość na wczesne znalezienie błędów mających wpływ na działanie aplikacji, a zarazem także na końcowy wygląd dokumentacji. 

\subsection{Organizacja dokumentu - podstawowe elementy planu projektu}

\begin{enumerate}
    \item \textbf{Wstęp} - Rozdział określający, cele, rolę i zakres dokumentacji.
    \item \textbf{Definicje} - Rozdział ten jest słownikiem pojęć i definicji wykorzystywanych w projekcie.
    \item \textbf{Zakres} - Rozdział zawiera zwięzłe streszczenie celów projektu i krótki opis tworzonego produktu.
    \item \textbf{Produkt projektu} - Rozdział w którym należy wymienić i opisać wszystkie produkty(systemy, instalacje, dokumentacja) zakładanego projektu.
    \item \textbf{Model procesu projektowego}
    \item \textbf{Organizacja projektu}
    \item \textbf{Zarządzanie}
    \item \textbf{Proces techniczny}
    \item \textbf{Etapy pracy, harmonogram i budżet}
\end{enumerate}

\subsection{Dokumenty powiązane / załączniki}
Brak dodatkowych dokumentów

\newpage

\section{Definicje}

Słownik pojęć i definicji wykorzystywanych w projekcie, oraz w obrębie tej dokumentacji. 

\begin{table}[htb]
\centering
  \begin{tabular}{c|c}
  \hline
  {\bf Pojęcie} & {\bf Definicja} \\
  \hline
  \textbf{Aplikacja webowa} & Tutaj definicja\\
  \hline
  \end{tabular}
\caption{Słownik pojęć}
\label{tab:slownik}
\end{table}

\newpage

\section{Zakres}

\end{document}
