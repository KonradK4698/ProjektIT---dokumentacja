\documentclass[12pt,a4paper]{article}
\usepackage[polish]{babel}
\usepackage[T1]{fontenc}
\usepackage[utf8x]{inputenc}
\usepackage{hyperref}
\usepackage{url}
\usepackage{graphicx}

\addtolength{\hoffset}{-1.5cm}
\addtolength{\marginparwidth}{-1.5cm}
\addtolength{\textwidth}{3cm}
\addtolength{\voffset}{-1cm}
\addtolength{\textheight}{2.5cm}
\setlength{\topmargin}{0cm}
\setlength{\headheight}{0cm}

\title{Dokumentacja projektu}
\author{Kmak Konrad}
\date{\today}

\begin{document}

\maketitle

\newpage

\tableofcontents

\newpage

\section{Wstęp}
\subsection{Cel dokumentacji}

Plan projektu jest niezbędnym elementem w procesie projektowania aplikacji. Nieustannie zmieniająca się sytuacja na rynku wymaga, aby każdy projekt został odpowiednio przygotowany, a także przeanalizowany. 

Celem niniejszego dokumentu jest określenie niezbędnych elementów, pozwalających na utworzenie spójnej, poprawnej, oraz w pełni zgodnej z wymaganiami klienta aplikacji. Dokładne przygotowanie planu, umożliwia dostrzeżenie, a następnie korektę możliwych zagrożeń które mogłyby znacząco wpłynąć na efekt końcowy, skutkując stratami finansowymi, lub brakiem możliwości ukończenia projektu. Dokument ten można traktować jako harmonogram projektu, przedstawiający przybliżony czas realizacji, a także plan pracy obligujący do wykonania wskazanych w nim zadań. 

Plan projektu jest dokumentem formalnym, określającym główne elementy związane z procesem powstawania aplikacji, dzięki czemu może zostać wykorzystany jako dokument potwierdzający jakość wykonanego oprogramowania, a także być podstawą do potwierdzenia, czy klient oraz wykonawca w pełni wywiązali się z warunków zawartych w umowie.

\subsection{Zakres dokumentacji}

Zakres niniejszej dokumentacji przedstawia informacje na temat planowania, analizy, zarządzania oraz kosztów tworzonej aplikacji. Początkowe informacje pozwalają na zapoznanie się z przeznaczeniem dokumentu oraz jego głównym zastosowaniem. Przechodząc do głównej części dokumentacji, można zauważyć że jest w pełni poświęcona procesowi planowania oraz zarządzania aplikacją. Zawarte są w niej takie aspekty jak: zakres projektu, czy model procesu projektowego. W tej części przedstawione są także informacje na temat zarządzania projektem, takie jak: 
\begin{itemize}
    \item podział struktury organizacyjnej,
    \item metody komunikacji, 
    \item cele i priorytety, 
    \item ryzyko i jego kontrola.
\end{itemize}
Końcowa część dokumentacji określa wymagania techniczne oraz harmonogram. Zawarte tam informacje, przybliżają w jakiej metodologii zostanie wykonany projekt, wraz z przedstawieniem narzędzi wykorzystywanych do utworzenia aplikacji. Ostatni punkt planu w pełni poświęcony jest harmonogramowi, który przedstawia etapy projektu, wraz z zadaniami oraz przyporządkowanymi do nich zasobami. Wraz z harmonogramem przedstawiony jest budżet, w którym zaprezentowany jest przewidywany kosz pracy nad aplikacją.

\subsection{Przeznaczenie dokumentu}

Najważniejszą osobą odpowiedzialną za utworzenie, a następnie aktualizację i utrzymanie niniejszej dokumentacji jest kierownik projektu. Kierownik jest wyspecjalizowanym członkiem zespołu, posiadającym szeroką wiedzę z zakresu zarządzania projektami, co czyni go osobą odgrywającą najważniejszą role w procesie powstawania aplikacji. Kierownik wykorzystując swoje doświadczenie zawodowe, a także odpowiednie przygotowanie merytoryczne, buduje zespół, prowadzi rozmowy z osobami związanymi z projektem, a następnie wykonuje dokumentację, która jest elementem niezbędnym do osiągnięcia sukcesu, za który odpowiada w trakcie trwania całego projektu.

Czynnym uczestnikiem w trakcie powstawania dokumentacji jest także zespół wykonawczy, odpowiedzialny za implementację aplikacji. Poprawna komunikacja pomiędzy kierownikiem projektu, a zespołem skutkuje doborem odpowiednich technologii, harmonogramem dostosowanym do możliwości pracowników, a także zmniejszeniem ryzyka przeszacowania czasu projektu.

Osobą mającą wpływ na końcowy wygląd planu projektu, jest również klient. Osoba zamawiająca aplikację, powinna jasno określić swoje wymagania dotyczące wizji projektu, pozwalając na utworzenie dokładnego harmonogramu. Klient sprawdzając w trakcie powstawania dokumentacji, czy jego koncepcja aplikacji pokrywa się z wizją kierownika projektu (którą określił w zakresie planu projektu), daje możliwość na wczesne znalezienie błędów mających wpływ na działanie aplikacji, a zarazem także na końcowy wygląd dokumentacji. 

\subsection{Organizacja dokumentu - podstawowe elementy planu projektu}

\begin{enumerate}
    \item \textbf{Wstęp} - Rozdział określający, cele, rolę i zakres dokumentacji.
    \item \textbf{Definicje} - Rozdział ten jest słownikiem pojęć i definicji wykorzystywanych w projekcie.
    \item \textbf{Zakres} - Rozdział zawiera zwięzłe streszczenie celów projektu i krótki opis tworzonego produktu.
    \item \textbf{Produkt projektu} - Rozdział w którym należy wymienić i opisać wszystkie produkty(systemy, instalacje, dokumentacja) zakładanego projektu.
    \item \textbf{Model procesu projektowego}
    \item \textbf{Organizacja projektu}
    \item \textbf{Zarządzanie}
    \item \textbf{Proces techniczny}
    \item \textbf{Etapy pracy, harmonogram i budżet}
\end{enumerate}

\subsection{Dokumenty powiązane / załączniki}
Brak dodatkowych dokumentów

\newpage

\section{Definicje}

Słownik pojęć i definicji wykorzystywanych w projekcie, oraz w obrębie tej dokumentacji. 

\begin{table}[htb]
\centering
  \begin{tabular}{c|c}
  \hline
  {\bf Pojęcie} & {\bf Definicja} \\
  \hline
  \textbf{Aplikacja webowa} & Tutaj definicja\\
  \hline
  \end{tabular}
\caption{Słownik pojęć}
\label{tab:slownik}
\end{table}

\newpage

\section{Zakres}

Zakres jest nieodłącznym elementem planu zarządzania projektem. Zawartość tego rozdziału określa niezbędne czynności które należy wykonać w celu osiągnięcia zamierzonych wyników. Poprawnie zdefiniowany, oraz wnikliwie przeanalizowany zakres projektu, wpływa pozytywnie na jakość tworzonego oprogramowania, pozwala na dokładniejsze oszacowanie kosztów, a także ułatwia zarządzanie zasobami. Rozdział ten przedstawia także cel oraz opis projektu, czyli kluczowe elementy wpływające na efekt końcowy tworzonej aplikacji.  

Głównym celem projektowanej aplikacji społecznościowej, jest utworzenie nowego miejsca w Internecie, pozwalającego na łatwiejsze nawiązywanie kontaktów wśród osób z szeroko pojętego świata IT, posiadających podobne zainteresowania, pasje oraz umiejętności. 

Projektowana aplikacja społecznościowa, implementowana będzie w formie aplikacji webowej. Użytkownik powinien posiadać dostęp do aplikacji za pomocą dowolnej przeglądarki internetowej, posiadanej na swoim urządzeniu. Podstawową niezbędną funkcjonalnością programu jest możliwość utworzenia nowego użytkownika poprzez formularz rejestracyjny. Każda zarejestrowana osoba, za pomocą formularza umożliwiającego zalogowanie do systemu, otrzyma dostęp do swojego nowo utworzonego konta, jego personalizacji, oraz możliwość skorzystania ze wszystkich funkcjonalności aplikacji przeznaczonych dla użytkownika. Personalizacja konta powinna odbywać się za pomocą wieloetapowych formularzy, pozwalających na podanie m.in. takich informacji jak: 
\begin{itemize}
    \item nazwa użytkownika,
    \item zainteresowania, 
    \item opis, 
    \item cele zawodowe, 
    \item posiadane umiejętności.
\end{itemize}
Po poprawnym uzupełnieniu formularza, użytkownik będzie posiadał pełen dostęp do aplikacji oraz przygotowanych w niej funkcjonalności. Każdy użytkownik aplikacji będzie mógł rozpocząć poszukiwanie nowych kontaktów, przeglądając oraz filtrując według własnych preferencji katalog użytkowników. W trakcie przeglądania katalogu użytkownik będzie mógł "odwiedzić" profil innego użytkownika, i zobaczyć udostępnione informacje o nim. Dodatkową opcją pozwalającą na spontaniczne rozpoczęcie rozmowy, będzie funkcjonalność umożliwiająca wylosowanie dowolnej, aktualnie aktywnej osoby. W celu umożliwienia nawiązania kontaktu pomiędzy użytkownikami, w aplikacji zostanie zaimplementowany komunikator pozwalający na rozpoczęcie rozmowy bez użycia dodatkowych aplikacji. 

Cele w projekcie jakie należy osiągnąć to: 
\begin{itemize}
    \item Zaprojektowanie strony głównej wraz z implementacją,
    \item Wykonanie projektu strony rejestracyjnej wraz z implementacją, 
    \item Zaprojektowanie strony pozwalającej na zalogowanie się do aplikacji, oraz jej implementacja,
    \item Opracowanie zawartości formularza odpowiadającego za personalizację konta, wykonanie projektu graficznego tego formularza, oraz jego implementacja,
    \item Utworzenie projektu profilu użytkownika, wraz z implementacją, 
    \item Wykonanie katalogu użytkowników, 
    \item Opracowanie funkcjonalności losującej użytkowników, 
    \item Wprowadzenie komunikatora do aplikacji. 
\end{itemize}

Warunkiem koniecznym wywiązania się z umowy pomiędzy klientem a firmą jest dostarczenie klientowi aplikacji kompletnej oraz poprawnie działającej. W tym celu zostaną określone zamierzone wyniki końcowe, pozwalające na określenie poprawności działania poszczególnych funkcjonalności w aplikacji. Efektami końcowymi, oczekiwanymi w trakcie działania aplikacji są: 
\begin{itemize}
    \item Możliwość przejścia do strony z formularzem rejestrującym, wraz z możliwością utworzenia nowego użytkownika
    \item Możliwość otworzenia strony z formularzem logującym do aplikacji, wraz z możliwością zalogowania użytkownika na jego konto,
    \item Po zalogowaniu nowego użytkownika, powinien zostać wyświetlony formularz pozwalający na personalizację konta, 
    \item Poprawne uzupełnienie formularza personalizującego powinno skutkować odblokowaniem dostępu do pozostałych funkcjonalności aplikacji,
    \item Użytkownik korzystający z katalogu użytkowników, powinien mieć możliwość korzystania z wszystkich dostępnych opcji filtrowania, oraz wyszukiwania użytkowników, 
    \item Funkcjonalność losująca użytkowników, powinna proponować tylko i wyłącznie użytkowników aktualnie aktywnych, 
    \item Komunikator powinien pozwalać na swobodną wymianę wiadomości w czasie rzeczywistym.
\end{itemize}

Na podstawie wyżej określonych celów, w tabeli poniżej został przedstawiony szacunkowy budżet wraz z prezentacją niezbędnych zasobów wykorzystywanych w trakcie pracy nad aplikacją. Tabela poniżej przedstawia jedynie oszacowanie budżetu wraz z wykorzystanymi zasobami, nie uwzględniając innych kosztów niż główne cele projektu. Dokładny budżet zostanie przedstawiony w rozdziale 9.

\begin{table}[htb]
\centering
  \begin{tabular}{c|c|c}
  \hline
  {\bf Cel} & {\bf Zasoby} & {\bf Koszt} \\
  \hline
  \textbf{Utworzenie strony głównej} & Zasób & Koszt\\
  \hline
  \textbf{Utworzenie rejestracji} & Zasób & Koszt\\
  \hline
  \textbf{Utworzenie logowania} & Zasób & Koszt\\
  \hline
  \textbf{Implementacja formularza personalizującego} & Zasób & Koszt\\
  \hline
  \textbf{Wprowadzenie profilu użytkownika} & Zasób & Koszt\\
  \hline
  \textbf{Wykonanie katalogu użytkownika} & Zasób & Koszt\\
  \hline
  \textbf{Opracowanie funkcji losującej} & Zasób & Koszt\\
  \hline
  \textbf{Implementacja komunikatora} & Zasób & Koszt\\
  \hline
  \end{tabular}
\caption{Tabela zasobów oraz budżetu}
\label{tab:zasoby}
\end{table}

\newpage

\section{Produkty projektu}

Produktem w projekcie nazywamy wszystkie policzalne dobra i usługi wytworzone w trakcie trwania projektu. Przykładowymi produktami występującymi w obrębie projektu są: opracowane oprogramowania, zainstalowane systemy, utworzone dokumentacje, a także przeprowadzone szkolenia pracowników. Produkty występujące w projekcie, scharakteryzowane są z wykorzystaniem opisu produktu który zawiera takie informacje jak: 
\begin{itemize}
    \item Nazwa produktu,
    \item Data i miejsce dostawy produktu, 
    \item Wymagana wielkość produktu, 
    \item Opis przeznaczenia oraz funkcji produktu, 
    \item Określenie elementów będących wyznacznikiem jakości produktu.
\end{itemize}

\newpage
Oczekiwanymi produktami w trakcie trwania projektu są: 

\begin{enumerate}
    \item \textbf{Aplikacja społecznościowa}
    \begin{table}[htb]
    \centering
      \begin{tabular}{|c|c|}
      \hline
      \multicolumn{2}{|c|}{Aplikacja społecznościowa - opis} \\
      \hline
      {\bf Nazwa} & {\bf Wartość} \\
      \hline
      \textbf{Nazwa produktu} & Aplikacja społecznościowa \\
      \hline
      \textbf{Data i miejsce dostawy} & ? \\
      \hline
      \textbf{Wielkość produktu} & ? \\
      \hline
      \textbf{Opis} & Aplikacja webowa umożliwiająca nawiązywanie nowych kontaktów.  \\
      \hline
      \textbf{Funkcje} 
                      & Rejestracja w aplikacji, \\
                      & Logowanie do aplikacji, \\
                      & Utworzenie profilu użytkownika, \\
                      & Wybór lub wylosowanie użytkownika w celu nawiązania kontaktu, \\
                      & Rozmowa z wykorzystaniem komunikatora. \\
      
      \hline
      \textbf{Wyznaczniki jakości} 
                      & Łatwość obsługi, \\
                      & Przejrzystość, \\
                      & Możliwość używania na dowolnej przeglądarce, \\
                      & Wysoka szybkość działania, \\
                      & Możliwość obsługi wielu użytkowników. \\
      
      \hline
      \end{tabular}
    \caption{Opis produktu ,,Aplikacja społecznościowa''}
    \label{tab:produkt1}
    \end{table}
    
    \item \textbf{Raporty dokumentujące ukończenie pracy nad funkcjonalnością}
    \begin{table}[htb]
    \centering
      \begin{tabular}{|p{0.35\linewidth} | p{0.6\linewidth}|}
      \hline
      \multicolumn{2}{|c|}{Raporty ukończonych funkcjonalności - opis} \\
      \hline
      {\bf Nazwa} & {\bf Wartość} \\
      \hline
      \textbf{Nazwa produktu} & Raport dokumentujący ukończoną funkcjonalność \\
      \hline
      \textbf{Data i miejsce dostawy} & ? \\
      \hline
      \textbf{Wielkość produktu} & ? \\
      \hline
      \textbf{Opis} & Dokument przedstawiający postępy wykonane w obrębie zaplanowanych funkcjonalności.  \\
      \hline
      \textbf{Funkcje} 
                      & Dokumentowanie postępów, \\
                      & Przedstawienie działania funkcjonalności, \\
      
      \hline
      \textbf{Wyznaczniki jakości} 
                      & Prosty przekaz, \\
                      & Zrozumiałe opisy, \\
                      & Terminowość, \\
      
      \hline
      \end{tabular}
    \caption{Opis produktu ,,raporty ukończonych funkcjonalności''}
    \label{tab:produkt2}
    \end{table}
    \newpage
    \item \textbf{Raport końcowy}
    \begin{table}[htb]
    \centering
      \begin{tabular}{|p{0.35\linewidth} | p{0.6\linewidth}|}
      \hline
      \multicolumn{2}{|c|}{Raporty końcowy - opis} \\
      \hline
      {\bf Nazwa} & {\bf Wartość} \\
      \hline
      \textbf{Nazwa produktu} & Raport końcowy \\
      \hline
      \textbf{Data i miejsce dostawy} & ? \\
      \hline
      \textbf{Wielkość produktu} & ? \\
      \hline
      \textbf{Opis} & Dokument zawierający podsumowanie projektu.  \\
      \hline
      \textbf{Funkcje} 
                      & Przedstawienie wykonanych celów, \\
                      & Podsumowanie kosztów, \\
                      & Podsumowanie zasobów, \\
                      & Możliwość oceny projektu przed jego zakończeniem \\
      \hline
      \textbf{Wyznaczniki jakości} 
                      & Dokładne opisy poczynionych działań, \\
                      & Szczegółowe przedstawienie kosztów, \\
                      & Terminowość, \\
      \hline
      \end{tabular}
    \caption{Opis produktu ,,raport końcowy''}
    \label{tab:produkt3}
    \end{table}
\end{enumerate}

\newpage



\end{document}
